%Dokumentenklasse, andere Beispiele: Statt llncs "`article"' "`report"', "`book"'...
%Achtung llncs ben�tigt die Datei llncs.cls im gleichen Ordner
\documentclass[runningheads]{llncs}

%sorgt f�r deutsche �bersetzungen, z.B. "`Literatur"' statt "`References"'
\usepackage{ngerman}
%Erm�glicht das Einbinden von Bildern
\usepackage{graphicx}
%Darstellung deutscher Sonderzeichen (Umlaute etc)
\usepackage[ansinew]{inputenc}
%Darstellung von URLs
\usepackage{url}

\usepackage{amsmath}
\usepackage{amsfonts}
\usepackage{amssymb}
\usepackage{amstext}

%Spezielle Einbindung einer URL zur Angabe in der caption
\urldef \urlSchnab \url{http://upload.wikimedia.org/wikipedia/commons/f/f2/Platypus.jpg}

\begin{document}

%Ben�tigte Angaben f�r die Titelseite
\title{\\ Automatisierung von Abrechnungsprozessen mit BPMN in Bezug auf Crowdcode GmbH \& Co.KG}
%Hier Zeilenumbruch durch \\, da \newline in dieser Umgebung nicht funktioniert.
\author{Nooshin Naghavi \\ Matrikel-Nr: 2401101}
%Hier das Institut angeben
\institute{Schnabeltier Universit�t}

%Erstellung des Titels
\maketitle

\begin{abstract}
Sehr kurze Zusammenfassung (1-3 S�tze) des Textes in welchen Aufgabe, vorgehensweise und Ergebnis vorgestellt werden.
\end{abstract}

\section{Einleitung}
Um die L\"ucke zwischen Managemnet und IT zu schlie�en, setzen Unternehmen seit Jahren eine Vielfallt an Werkzeugen und Notationen f\"ur die Modelierung von Gesch\"aftsprozessen ein. Das hei�t, dass die Gesch\"aftprozessen formal definiert werden m\"ussen, um sie zu dokumentieren, verstehen, automatisieren und verbessern. Eine von der h\"aufig benutzten Notationen ist BPMN.


%Erster Abschnitt, in geschweiften Klammern Titel des ersten Abschnittes
\section{Hintergrund}

die monatlichen Rechnungen von Subunternehmen und Dienstleister an Crowdcode m\"ussen mit Zeitdaten abgeglichen werden und entsprechend die Zahlungen freigegeben werden. Belege m\"ussen an Steuerberater weitergeleitet und verschl\"usselt archiviert werden. Crowdcode hat vor, diese Prozessabl\"aufe zu automatisieren und \"uber verschiedene Systeme bereitzustellen. 


\section{Zielsetzung}

Die Arbeit sollte diese M\"oglichkeit darstellen, wie die BPMN Prozesse mit Camunda BPM Platform modelliert werden k\"onnen. Dabei werden ein oder mehrere Gesch\"aftsprozesse analysiert und modelliert. Nach der Modellierung wird eine Auswertung der m\"oglichen L\"osungsans\"atze zur Umsetzung erstellt. Unter Ber\"ucksichtigung der Restzeit wird ein modellierter Prozessablauf durch realisiert.


\end{document}
