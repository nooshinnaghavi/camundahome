%Dokumentenklasse, andere Beispiele: Statt llncs "`article"' "`report"', "`book"'...
%Achtung llncs ben�tigt die Datei llncs.cls im gleichen Ordner
\documentclass[runningheads]{llncs}

%sorgt f�r deutsche �bersetzungen, z.B. "`Literatur"' statt "`References"'
\usepackage{ngerman}
%Erm�glicht das Einbinden von Bildern
\usepackage{graphicx}
%Darstellung deutscher Sonderzeichen (Umlaute etc)
\usepackage[ansinew]{inputenc}
%Darstellung von URLs
\usepackage{url}

\usepackage{amsmath}
\usepackage{amsfonts}
\usepackage{amssymb}
\usepackage{amstext}

\begin{document}

\begin{titlepage}
    \begin{center}
    \huge \textbf{\textsf{Automatisierung von Abrechnungsprozessen mit BPMN in Bezug auf Crowdcode GmbH \& Co.KG}} \\
    \vspace{2cm}
    \LARGE\textbf{\textsc{Expos\'e zur Bachelor-Thesis}}\\
    \vspace{1cm}
    \normalsize
    \today \\
    \vspace{2.5cm}
    \large \textbf{Rheinische Friedrich-Wilhelms-Universit\"at Bonn}\\
    \vspace{3cm}
    \end{center}
 \normalsize{
    \begin{tabular}{ll}
    	Name: & {Nooshin Naghavi} \\
    	Matrikelnummer: & {2401101} \\
    	Studiengang: & Informatik\\
      Erstbetreuer: & {Dr. G�nter Kniesel} \\
      Zweitbetreuer: & {Ingo Duppe} \\
    \end{tabular}\\
    }
\end{titlepage}

\section{Motivation}
Um die L\"ucke zwischen Managemnet und IT zu schlie�en, setzen Unternehmen seit Jahren eine Vielfallt an Werkzeugen und Notationen f\"ur die Modelierung von Gesch\"aftsprozessen ein. Das hei�t, dass die Gesch\"aftprozessen formal definiert werden m\"ussen, um sie zu dokumentieren, verstehen, automatisieren und verbessern. Eine von der h\"aufig benutzten Notationen ist BPMN.

\section{Zielsetzung und Grenzen}

\subsection{Ausgangssituation}

die monatlichen Rechnungen von Subunternehmen und Dienstleister an Crowdcode m\"ussen mit Zeitdaten abgeglichen werden und entsprechend die Zahlungen freigegeben werden. Belege m\"ussen an Steuerberater weitergeleitet und verschl\"usselt archiviert werden. Crowdcode hat vor, diese Prozessabl\"aufe zu automatisieren und \"uber verschiedene Systeme bereitzustellen. 

\subsection{Zielsetzung}

Die Arbeit sollte diese M\"oglichkeit darstellen, wie die BPMN Prozesse mit Camunda BPM Platform modelliert werden k\"onnen. Dabei werden ein oder mehrere Gesch\"aftsprozesse analysiert und modelliert. Nach der Modellierung wird eine Auswertung der m\"oglichen L\"osungsans\"atze zur Umsetzung erstellt. Unter Ber\"ucksichtigung der Restzeit wird ein modellierter Prozessablauf durch realisiert.

\subsection{Zielgruppe}

\subsection{Abgrenzungen}

Folgende Teile k\"onnen nicht in der Arbeit behandelt werden:

\section{Gliederung}

Das ist nur eine grobe Sch\"atzung und kann sich teilweise \"andern.
\begin{description}
\item
   \begin{itemize}
      \item Eidesstaattliche Erkl\"arung
			\item Vorwort
			\item Abstract
			\item Inhaltsverzeichnis
			\item Einleitung
      \begin{itemize}
         \item Grundlage
         \item Projektpartner
				 \item Motivation
      \end{itemize}
      \item Analyse und Modellierung
			\begin{itemize}
			   \item Anforderungsanalyse
				 \item Modellierung
      \end{itemize}
			\item Umsetzung
			\item Ergebnisse
			\item Zusammenfassung
			\item Glossar
			\item Literaturverzeichniss
   \end{itemize}
	\end{description}
	
	\section{Vorgehen}
	\section{Herausforderung}
	
	Vielleicht Einarbeitung
	
	\section{Zeitplan}
\end{document}
