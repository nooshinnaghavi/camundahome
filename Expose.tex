%Dokumentenklasse, andere Beispiele: Statt llncs "`article"' "`report"', "`book"'...
%Achtung llncs ben�tigt die Datei llncs.cls im gleichen Ordner
\documentclass[runningheads]{llncs}

%sorgt f�r deutsche �bersetzungen, z.B. "`Literatur"' statt "`References"'
\usepackage{ngerman}
%Erm�glicht das Einbinden von Bildern
\usepackage{graphicx}
%Darstellung deutscher Sonderzeichen (Umlaute etc)
\usepackage[ansinew]{inputenc}
%Darstellung von URLs
\usepackage{url}

\usepackage{amsmath}
\usepackage{amsfonts}
\usepackage{amssymb}
\usepackage{amstext}

\begin{document}

\begin{titlepage}
    \begin{center}
    \huge \textbf{\textsf{Automatisierung von Abrechnungsprozessen mit BPMN in Bezug auf Crowdcode GmbH \& Co.KG}} \\
    \vspace{2cm}
    \LARGE\textbf{\textsc{Expos\'e zur Bachelor-Thesis}}\\
    \vspace{1cm}
    \normalsize
    \today \\
    \vspace{2.5cm}
    \large \textbf{Rheinische Friedrich-Wilhelms-Universit\"at Bonn}\\
    \vspace{3cm}
    \end{center}
 \normalsize{
    \begin{tabular}{ll}
    	Name: & {Nooshin Naghavi} \\
    	Matrikelnummer: & {2401101} \\
    	Studiengang: & Informatik\\
      Erstbetreuer: & {Dr. G�nter Kniesel} \\
      Zweitbetreuer: & {Ingo Duppe} \\
    \end{tabular}\\
    }
\end{titlepage}

\section{Motivation}
Um die L\"ucke zwischen Managemnet und IT zu schlie�en, setzen Unternehmen seit Jahren eine Vielfallt an Werkzeugen und Notationen f\"ur die Modelierung von Gesch\"aftsprozessen ein. Das hei�t, dass die Gesch\"aftprozessen formal definiert werden m\"ussen, um sie zu dokumentieren, verstehen, automatisieren und verbessern. [1] Denn die Unternehmen nehmen gro�en Wert dran, ihre Gesch�ftsabl\"aufe zu automatisieren, weil dadurch viel Zeit und Kosten gespart werden kann, wird die Frage h\"aufig gestellt, mit welchen Werkzeugen diese Realisierung erm�glicht ist.\newline
Eine von der h\"aufig benutzten Notationen ist BPMN und die modellierten Prozessen mit BPMN lassen sich durch Camunda BPM automatisieren. Camunda BPM ist eine gute Wahl, weil dadurch Prozess-Kosten sich verringern, Prozess-Durchl�ufe sich beschleunigen, in vorhandener IT eibettbar ist, abh\"angig von keiner Hersteller ist, keine speziellen Entwickler ben\"otigt und viel mehr. [2]


\section{Zielsetzung und Grenzen}

\subsection{Ausgangssituation}

Das Unternehmen Crowdcode GmbH \& Co.KG arbeiten in verschiednen Projekten und sind ein ziemlich neu gegrundetes Unternehmen. Crowdcode w�nscht sich Abrechnungsprozessen automatisieren zu lassen. N\"amlich die monatlichen Rechnungen von Subunternehmen und Dienstleister m\"ussen mit Zeitdaten abgeglichen werden und die entsprechenden Zahlungen freigegeben werden. Belege m\"ussen an Steuerberater weitergeleitet, verschl\"usselt archiviert und \"uber verschiedene Systeme bereitgestellt werden.  

\subsection{Zielsetzung}

Ziel dieser Arbeit ist die BPMN Prozesse mit Hilfe von Camunda BPM engine zu realisieren. Dabei werden mehrere Werkzeuge eingesetzt. ......

\subsection{Zielgruppe}

Hauptnutzer der Anwendung werden die Mitarbeiter von Crowdcode sein. .....

\subsection{Abgrenzungen}

eine vollst�ndige Analyse, Modellierung und Umsetzung von Abrechnungsprozessen kann nicht im Rahmen dieser Arbeit erf�llbar sein.

\section{Gliederung}

Die Gliederung kann im Laufe der Arbeit ge\"andert werden.

\begin{description}
\item
   \begin{itemize}
      \item Eidesstaattliche Erkl\"arung
			\item Vorwort
			\item Abstract
			\item Inhaltsverzeichnis
			\item Einleitung
      \begin{itemize}
         \item Grundlage
         \item Projektpartner
				 \item Motivation
      \end{itemize}
      \item Analyse und Modellierung
			\begin{itemize}
			   \item Anforderungsanalyse
				 \item Modellierung
      \end{itemize}
			\item Umsetzung
			\item Ergebnisse
			\item Zusammenfassung
			\item Glossar
			\item Literaturverzeichniss
   \end{itemize}
	\end{description}
	
	\section{Vorgehen}
	
	Die Arbeit beinhaltet ...Abschnitte. ZUerst wird... Danach... Zum Schluss
	
	\section{Herausforderung}
	
	
	\section{Zeitplan}
	
	ob es n\"otig ist!
	
\begin{thebibliography}{3}

\bibitem[1]{1}
computer society, \url{http://www.computer.org/web/chapters/Carlos-Monsalve}, abgerufen am 27.12.2015.

\bibitem[2]{2}
Camunda, \url{https://camunda.com/}, abgerufen am 27.12.2015.

\end{thebibliography}
	
\end{document}
