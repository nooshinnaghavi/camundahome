%Dokumentenklasse, andere Beispiele: Statt llncs "`article"' "`report"', "`book"'...
%Achtung llncs ben�tigt die Datei llncs.cls im gleichen Ordner
\documentclass[runningheads]{llncs}

%sorgt f�r deutsche �bersetzungen, z.B. "`Literatur"' statt "`References"'
\usepackage{ngerman}
%Erm�glicht das Einbinden von Bildern
\usepackage{graphicx}
%Darstellung deutscher Sonderzeichen (Umlaute etc)
\usepackage[ansinew]{inputenc}
%Darstellung von URLs
\usepackage{url}
\usepackage{amsmath}
\usepackage{amsfonts}
\usepackage{amssymb}
\usepackage{amstext}
\usepackage{fancyhdr}
\pagestyle{fancy}
\fancyhf{}
\fancyhead[R]{\leftmark}
\fancyhead[L]{\rightmark}
\fancyfoot[C]{\thepage}
\setlength{\headheight}{15pt}
\renewcommand{\sectionmark}[1]{\markboth{#1}{}}


\begin{document}
\begin{titlepage}
    \center
    \begin{flushright}
			\includegraphics[scale=0.5]{logo.png}\\[1cm]
		\end{flushright}
    \huge \textbf{\textsf{Expos\'e zur Bachelor-Thesis}} \\
    \vspace{2cm}
    \large\textbf{\textsc{Automatisierung von Abrechnungsprozessen mit BPMN in Bezug auf Crowdcode GmbH \& Co.KG}}\\
    \vspace{1cm}
    \normalsize
		Verfasser: Nooshin Naghavi\\
		Martikelnummer: 2401101\\
		\vfill
		\begin{minipage}{0.9\textwidth}
    \begin{flushleft}
     1. Betreuer: Dr. G�nter Kniesel\\
     2. Betreuer: Ingo D\"uppe
    \end{flushleft}
    \end{minipage}
		\vfill
		\today
\end{titlepage}

\section{Motivation (Problemstellung)}
Um die L\"ucke zwischen Management und IT zu schlie�en, setzen Unternehmen eine Vielfallt von Werkzeugen und Notationen zur Optimierung von Gesch\"aftsprozessen ein. Die Unternehmen legen dabei gro�en Wert darauf, ihre Gesch�ftsabl\"aufe weitesgehend zu automatisieren, weil sie dadurch nicht nur viel Zeit und Kosten sparen k�nnen, sondern die Prozesse dadurch auch weniger fehleranf�llig sind. Dazu m�ssen die Gesch\"aftprozesse zun\"achst formal definiert werden um sie dann entsprechend dokumentieren, verstehen, automatisieren und verbessern zu k�nnen\newline
Eine der dazu am h\"aufigsten verwendeten Notationen ist BPMN. Die mit BPMN modellierten Prozessen k\"onnen mit Hilfe von Camunda BPM umgesetzt werden. Camunda ist ein offenes Framework, das in der vorhandenen technischen Umgebung nahtlos eingebettet werden kann. Es erlaubt die Nutzung des Kompletten Java--\"Okosystems f\"ur die Entwicklung von Prozessanwendungen und macht keine Einschr\"ankung hinsichtlich der Verwendung anderer Komponenten und Frameworks (z.B. Spring, Java EE etc.). [1]

\newpage

\section{Zielsetzung und Grenzen}

\subsection{Ausgangssituation}

Das Unternehmen Crowdcode GmbH \& Co. KG ist ein neu gegr\"undetes Unternehmen, das an verschiedenen Projekten arbeitet. Crowdcode w�nscht sich seine Abrechnungsprozesse automatisieren zu lassen. Und zwar sollen die monatlichen Rechnungen von Subunternehmen und Dienstleister mit den Zeitdaten abgeglichen werden und die entsprechenden Zahlungen freigegeben werden. Belege m\"ussen an Steuerberater weitergeleitet, verschl\"usselt archiviert und \"uber verschiedene Systeme bereitgestellt werden.  

\subsection{Zielsetzung}

Ziel dieser Arbeit ist es einen konkreten BPMN Prozess mit Hilfe von Camunda BPM engine zu realisieren. Dabei m\"ochte ich besonderes auf Analyse und Modellierung von Prozessabl\"aufe eingehen.

\subsection{Zielgruppe}

Hauptnutzer der Anwendung werden die Mitarbeiter von Crowdcode sein. 

\subsection{Abgrenzungen}

eine vollst�ndige Analyse, Modellierung und Umsetzung von Abrechnungsprozessen ist im Rahmen dieser Arbeit nicht vorgesehen.

\newpage

\section{Gliederung}

Die Gliederung kann im Laufe der Arbeit ge\"andert werden.

\begin{itemize}
  \item Eidesstaattliche Erkl\"arung
  \item Inhaltsverzeichnis
  \item Abstract
  \item Einleitung
      \begin{itemize}
         \item Motivation und Problemstellung
         \item Zielsetzung
				 \item Aufbau der Arbeit 
      \end{itemize}
 \item Grundlagen
			\begin{itemize}
			   \item BPMN
				 \item Spring Framework
				 \item Camunda BPM
      \end{itemize}
 \item Prozesserfassung
 \item Ergebnisse
			\begin{itemize}
				\item Modellierung
				\item Umsetzung
			\end{itemize}
  \item Diskussion
			\begin{itemize}
				\item Auswertung der Ergebnis
				\item Vergleich zur alternativen M\"oglichkeiten
				\item Fazit
			\end{itemize}
	 \item Literaturverzeichnis
 \end{itemize}
	
	\newpage
	
	\section{Grobe Zeitplanung}
	
	\begin{itemize}
	  \item am 10.01.16 Abgabe des Expos\'e
		\item am 15.02.16 Anmeldung zur Bachelorarbeit
		\item Ende Februar
		\begin{itemize}
		  \item Die Gliederung und Inhalte festlegen
			\item Einleitung ausarbeiten
			\item Technologien zur Umsetzung finden
		\end{itemize}
		\item Mitte M\"arz 
		\begin{itemize}
			\item Processerfassung
			\item Processmodellierung
			\item Implementierung der Weboberfl\"ache
		\end{itemize}
		\item Mitte April ...
		\item Mitte Mai ...
		\item Summe Bearbeitungszeiten = ca. X Wochen
		\item Abgabetermin
	\end{itemize}
	
	\newpage
	
	\section{Vorgesehene Quellen}
	
	
\begin{thebibliography}{1}

\bibitem[1]{1}
Camunda, \url{https://camunda.com/}, abgerufen am 27.12.2015.

\end{thebibliography}
	
\end{document}
